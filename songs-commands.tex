% For easy management of optional arguments
% usage: \ifempty{test}{value test is empty}{value otherwise}
\newcommand{\ifempty}[3]{\ifx#1\\\\#2\else#3\fi} 
% usage: \defaut{val}{default value if val is empty}
\newcommand{\defaut}[2]{\ifx#1\\\\#2\else#1\fi}

% Song elements
\newcommand*{\songTitle}[1]{\begin{center} \textbf{\Huge #1} \end{center}}
\newcommand*{\artist}[1]{\textsc{\large #1}}
\newcommand*{\originalArtist}[1]{\small\emph{Original artist: \textsc{#1}}}

% Song environment
% usage: \begin{Song}[titleForIndex]{title}[artistForIndex]{artist}[originalArtist][originalArtistForIndex]
\newenvironmentx{Song}[6][1,3,5,6]{%
	\newpage
	\fancyhead[R]{
		\artist{#4}
		\ifempty{#5}{}{\\ \originalArtist{#5}}
	}
	\songTitle{#2}
	\index{\defaut{#1}{#2}@ #2}          % indexes the title
	\index{\defaut{#3}{#4}@ \textbf{#4}} % indexes the artist
	\index{\defaut{#6}{#5}@ \textbf{#5}} % indexes the original artist
	\begin{center}
}{%
	\end{center}
}

% Paratext environment
\newenvironment{Paratext}{
	\begin{minipage}{\textwidth}
		\begin{it}
}{%
		\end{it}
	\end{minipage}
	\vspace{0.2cm}
}

% Chords grids
\newenvironment{Chords}[1][]{
	\begin{center}
	\ifempty{#1}{}{#1\\\vspace{0.2cm}}
	\begin{tabular}{|c|c|c|c|c}
}{%
	\end{tabular}
	\end{center}
}

% Colors
\definecolor{maleColor}{HTML}{271dca}
\definecolor{femaleColor}{HTML}{d2149f}

% Lyrics environments
\newenvironment{Lyrics}   {\obeylines}{}
\newenvironment{Verse}    {\begin{Lyrics}}{\end{Lyrics}}
\newenvironment{PreChorus}{\begin{Lyrics}}{\end{Lyrics}}
\newenvironment{Bridge}   {\begin{Lyrics}}{\end{Lyrics}}
\newenvironment{Chorus}   {\begin{Lyrics}\begin{bf}}{\end{bf}\end{Lyrics}}

% Colored lyrics environments
% usage: \begin{Singer}{color}{singerName} (name not used for now)
\newenvironment{Singer}[2]{\begin{color}{#1}}{\end{color}}
\newenvironment{SingerMan}[1][]{\begin{Singer}{maleColor}{#1}}{\end{Singer}}
\newenvironment{SingerWoman}[1][]{\begin{Singer}{femaleColor}{#1}}{\end{Singer}}

% Inline colored lyrics
\newcommand*{\singer}[3]{{\color{#1}#3}}
\newcommand*{\singerMan}[2][]{\singer{maleColor}{#1}{#2}}
\newcommand*{\singerWoman}[2][]{\singer{femaleColor}{#1}{#2}}

% Space management
\newcommand*{\espaceInterStrophe}{\vspace{0.5cm}}
\newcommand*{\espaceInterGrille}{\vspace{0.0cm}}
\newcommand*{\espaceTexteVersGrille}{\vspace{1cm}}
\newcommand*{\espaceParatexteVersSimpleColonne}{\vspace{0.6cm}}
\newcommand*{\espaceParatexteVersMultiColonnes}{}

% Mid-lyrics indications
\newcommand*{\direction}[2][]{\textbf{[\textsc{#2}\ifempty{#1}{}{ (#1)}]}}
\newcommand*{\indication}[1]{\textbf{[#1]}}
\newcommand*{\indicleft}[1]{{\raggedright\indication{#1}\\}}
\newcommand*{\intro}[1][]{\direction[#1]{Intro}}
\newcommand*{\aurefrain}[1][]{\direction[#1]{Refrain}}
\newcommand*{\tochorus}[1][]{\direction[#1]{Chorus}}
\newcommand*{\bis}{(\emph{bis})}
\newcommand*{\adlib}{(\emph{ad lib\dots})}

% Special lyrics text behavior
\newcommand*{\backup}[1]{\emph{(#1)}}
\newcommand*{\rejet}[1]{\\ {\raggedleft[#1\par}}

% Chords grids macros (use in this order)
\newcommand{\diese}{$^\sharp$}
\newcommand{\bemol}{$^\flat$}
\newcommand{\mineur}{m}
\newcommand{\sept}{$^7$}
\newcommand{\neuf}{9}
\newcommand{\majsept}{$^{7\text{M}}$}
\newcommand{\neufbemol}{\neuf$^{(\flat 9)}$}
\newcommand{\neufaug}{$^{(\sharp 9)}$}
\newcommand{\quinteaug}{$^{(\sharp 5)}$}
\newcommand{\quintedim}{$^{(\flat 5)}$}
\newcommand{\addtreiziemedim}{add$\flat 13$}
\newcommand*{\sus}[1]{sus{#1}}
\newcommand*{\add}[1]{add{#1}}
\newcommand{\puissance}{5}                    % 1 5
\newcommand{\diminue}{\emph{dim}}             % 1 3m 5b 6
\newcommand{\demidim}{\mineur\sept\quintedim} % 1 3m 5b 7m (we could use $\varnothing$)

\newcommand{\gridGroupTitle}[1]{\textsc{#1}}
\newcommand{\gridGroupNormal}{\gridGroupTitle{Normal}}
\newcommand{\gridGroupCapo}[1]{\gridGroupTitle{Normal capo #1}}
\newcommand{\gridGroupBetter}{\gridGroupTitle{Better}}
\newcommand{\gridGroupBetterCapo}[1]{\gridGroupTitle{Better capo #1}}
\newcommand{\gridGroupPerformer}[1]{\gridGroupTitle{#1}}
\newcommand{\gridGroupPerformerCapo}[2]{\gridGroupTitle{#1 (Capo #2)}}
\newcommand{\gridGroupNormalAndBetter}[1]{\gridGroupTitle{Normal (Better capo #1)}}
\newcommand{\gridGroupCapoAndBetter}[2]{\gridGroupTitle{Normal capo #1 (Better capo #2)}}
\newcommand{\gridGroupCapoAndBetterWithout}[1]{\gridGroupTitle{Normal capo #1 (Better without)}}