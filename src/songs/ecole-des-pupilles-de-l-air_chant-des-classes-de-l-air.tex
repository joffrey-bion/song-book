\begin{Song}{Chant des Classes de l'Air}{Ecole des Pupilles de l'Air}

\begin{Verse}
Jeune français, la passion à nos portes t’a conduit
Depuis toujours les rêves emplissent ton esprit
Mais découvrant bientôt la rigueur militaire
Tu fais tes premiers pas aux Pupilles de l’Air
Tu ne seras plus jamais seul car on est tous unis
Te voilà à l’aube, à l’aube d’une nouvelle vie
Ensemble il vous faudra faire preuve de cohésion
Pour forger l’âme de la nouvelle promotion
\end{Verse}
\espaceInterStrophe

\begin{Chorus}
Poussin, Poussin ton rêve est de servir
Et en ces lieux écrit ton avenir
Poussin, mon frère, montre toi digne et fier
Et à jamais honore les Classes de l’Air
\end{Chorus}
\espaceInterStrophe

\begin{Verse}
Jour après jour la jeune promotion grandit
Guidée par ses aînés s’élève et s’anoblit
A force de persévérance et de rigueur
Comprend le rôle d’officier et ses valeurs
Elle affronte avec courage les difficultés
Faisant sienne la devise Faire Face pour tout donner
Honorant alors les plus belles traditions
Elle saura y puiser toute sa détermination
\end{Verse}
\espaceInterStrophe

\aurefrain\\
\espaceInterStrophe

\begin{Verse}
Parvenu à la place qui fût celle de tes aînés
Transmets avec passion les valeurs inculquées
Bientôt tes rêves se réalisent et tu t’envoles
Fidèle ambassadeur de l’esprit de l’Ecole
Transformé par ces années, poursuis ton chemin
Mais où que tu ailles, n’oublie jamais d’où tu viens
Grave en ton cœur ces longues années d’abnégation
Dans la joie, la douleur, ensemble nous combattrons
\end{Verse}
\espaceInterStrophe

\aurefrain

\vfill

\begin{multicols}{2}

\begin{Chords}[Couplet]
\hline
A\mineur & E         & A\mineur & G                             \\\hline
C        & G         & A\mineur & \slashbox[0pt][r]{E\mineur}{A\mineur} \\\hline
A\mineur & E\mineur  & E\mineur & A\mineur                      \\\hline
A\mineur & B\demidim & E        & A\mineur                      \\\hline
A\mineur & E\mineur  & E\mineur & E\mineur                      \\\hline
A\mineur & B\demidim & E        & A\mineur                      \\\hline
\end{Chords}
\columnbreak

\begin{Chords}[Refrain]
\hline
A\mineur        & E\mineur               & A\mineur & C                   \\\hline
\slashbox{C}{G} & \slashbox{A\mineur}{E} & A\mineur & \slashbox{F}{E (G)} \\\hline
A\mineur (C)                                                              \\\cline{1-1}
\end{Chords}

\end{multicols}
\vfill

\end{Song}



