\begin{Song}[Tristitude]{La Tristitude}{Oldelaf}

\begin{multicols}{2}
\begin{Verse}
La tristitude
C'est quand tu viens juste d'avaler un cure dent
Quand tu te rends compte que ton père est \rejet{suisse-allemand}
Quand un copain t'appelle pour son déménagement
Et ça fait mal
\espaceInterStrophe

La tristitude
C'est franchir le tunnel de Fourvière le 15 août
Quand tu dois aller vivre a Nogent le rottrou
Quadn ton coiffeur t'apprend que t'as des reflets \rejet{roux}
Et ça fait mal
\end{Verse}
\espaceInterStrophe

\begin{Chorus}
La tristitude
C'est toi, c'est moi, c'est nous, c'est quoi
C'est un peu la détresse
Dans le creux de nos voix
\espaceInterStrophe

La tristitude
C'est mmmh, c'est whou, c'est eux, c'est vous
C'est la vie qui te dit que ça va pas du tout
\end{Chorus}
\vfill
\columnbreak

\begin{Verse}
La tristitude
C'est quand t'es choisi pour être gardien au \rejet{Handball}
Quand t'es dans la mercose de la princesse de Galle
Quand samedi soir c'est Taffi qui joue sur Canal
Et ça fait chier
\espaceInterStrophe

La tristitude
C'est quand tu marche pied nu sur un tout petit Lego
C'est quand lors d'un voyage en Inde tu bois de l'eau
Quand ton voisin t'annonce qu'il se met au saxo
Et ça fait mal
\end{Verse}
\espaceInterStrophe

\aurefrain
\espaceInterStrophe

\begin{Verse}
La tristitude
C'est quand ton frère siamois t'annonce qu'il a \rejet{l'SIDA}
Quand ta femme fait de l'échangisme un peu sans toi
Quand des jeunes t'appellent monsieur pour la première fois
Et ça fait mal
\espaceInterStrophe

La tristitude
C'est devenir styliste mais pour Eddy Mitchel
C'est conjuguer bouillir au subjonctif pluriel
C'est faire les courses le samedi d'avant noël
Et ça fait mal
\end{Verse}
\espaceInterStrophe

\aurefrain

\end{multicols}
\vfill

\begin{Chords}[Couplet]
\hline
G\mineur & C\mineur\sept & F & B\bemol\\\hline
E\bemol & A\sept & D\sept & D\sept\\\hline
\end{Chords}
\espaceInterGrille

\begin{Chords}[Refrain]
\hline
\slashbox{C\mineur}{F} & \slashbox{B\bemol}{E\bemol} & \slashbox{C\mineur\sept}{D\sept} & \slashbox{G\mineur}{G\sept}\\\hline
\slashbox{C\mineur}{F} & \slashbox{B\bemol}{E\bemol} & \slashbox{C\mineur\sept}{A\sept} & D\sept\\\hline
\end{Chords}
\vfill
\vfill

\end{Song}

% ==============================================================================

