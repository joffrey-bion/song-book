\begin{Song}{Comme Toi}{Jean-Jacques Goldmann}
\begin{multicols}{2}
\begin{Verse}
Elle avait les yeux clairs
Et la robe en velours
A côté de sa mère
Et la famille autour
Elle pose un peu distraite
Au doux soleil de la fin du jour
\espaceInterStrophe

La photo n'est pas bonne
Mais l'on peut y voir
Le bonheur en personne
Et la douceur d'un soir
Elle aimait la musique
Surtout Schumann, et puis Mozart
\end{Verse}
\espaceInterStrophe

\begin{Chorus}
Comme toi, comme toi, comme toi
Comme toi, comme toi, comme toi
Comme toi que je regarde tout bas
Comme toi qui dors en rêvant à quoi?
Comme toi, comme toi, comme toi
\end{Chorus}
\espaceInterStrophe

\begin{Verse}
Elle allait à l'école au village d'en bas
Elle apprenait les livres
Elle apprenait les lois
Elle chantait les grenouilles
Et les princesses qui dorment au bois
\espaceInterStrophe

Elle aimait sa poupée
Elle aimait ses amis
Surtout Ruth et Anna
Et surtout Jérémie
Et ils se marieraient un jour
Peut-être à Varsovie
\end{Verse}
\espaceInterStrophe

\tochorus
\espaceInterStrophe

\indication{Modulation : +1 ton}
\begin{Verse}
Elle s'appelait Sarah
Elle n'avait pas huit ans
Sa vie, c'était douceur,
Rêves, et nuages blancs
Mais d'autres gens
En avaient décidé autrement
\espaceInterStrophe

Elle avait tes yeux clairs
Et elle avait ton âge
C'était une petite fille
Sans histoires et très sage
Mais elle n'est pas née
Comme toi ici et maintenant!
\end{Verse}
\espaceInterStrophe

\tochorus
\end{multicols}

\vfill

\begin{multicols}{2}

\begin{Chords}[Couplet]
\hline
D\mineur & D\mineur & G\mineur & G\mineur\\\hline
A\sus4 & A\sus4 & D\mineur & D\mineur\\\hline
\end{Chords}
\espaceInterGrille

\begin{Chords}[Refrain]
\hline
G\mineur & A\sept & D\mineur & D\mineur(\sus2)\\\hline
G\mineur & A\sept & D\mineur & D\mineur(\sus2)\\\hline
G\mineur & A\sept & D\mineur & G\mineur\\\hline
B\bemol & A\sept & D\mineur & D\mineur(\sus2)\\\hline
\end{Chords}
\columnbreak

\begin{Chords}[Couplet (modulation)]
\hline
E\mineur & E\mineur & A\mineur & A\mineur\\\hline
B\sus4 & B\sus4 & E\mineur & E\mineur\\\hline
\end{Chords}
\espaceInterGrille

\begin{Chords}[Refrain (modulation)]
\hline
A\mineur & B\sept & E\mineur & E\mineur(\sus2)\\\hline
A\mineur & B\sept & E\mineur & E\mineur(\sus2)\\\hline
A\mineur & B\sept & E\mineur & A\mineur\\\hline
C & B\sept & E\mineur & E\mineur(\sus2)\\\hline
\end{Chords}
\vfill
~

\end{multicols}

\vfill

\end{Song}

% ==============================================================================

