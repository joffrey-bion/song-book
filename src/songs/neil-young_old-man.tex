\begin{Song}{Old Man}{Neil Young}

\begin{Paratext}
The song was written for Louis Avala, the caretaker of the Northern California Broken Arrow Ranch, which Young purchased for \$350,000 in 1970. The song compares a young man's life to an old man's and shows that the young man has, to some extent, the same needs as the old one.

\og Louis took me for a ride in his blue Jeep. He gets me up there on the top side of the place [\dots] and he says, "Well, tell me, how does a young man like yourself have enough money to buy a place like this?" And I said, "Well, just lucky, Louie, just real lucky." And he said, "Well, that's the darndest thing I ever heard." And I wrote this song for him.\fg
\end{Paratext}

\begin{multicols}{2}

\begin{Verse}
Old man look at my life
I'm a lot like you were
\bis
\espaceInterStrophe

Old man look at my life
Twenty four
and there's so much more
Live alone in a paradise
That makes me think of two
\espaceInterStrophe

Love lost, such a cost
Give me things
that don't get lost
Like a coin that won't get tossed
Rolling home to you
\end{Verse}
\espaceInterStrophe

\begin{Chorus}
Old man take a look at my life
I'm a lot like you
I need someone to love me
the whole day through
Ah, one look in my eyes
and you can tell that's true
\end{Chorus}
\espaceInterStrophe

\begin{Verse}
Lullabies, look in your eyes
Run around the same old town
Doesn't mean that much to me
To mean that much to you
\espaceInterStrophe

I've been first and last
Look at how the time goes past
But I'm all alone at last
Rolling home to you
\end{Verse}
\espaceInterStrophe

\begin{Chorus}
Old man take a look at my life
I'm a lot like you
I need someone to love me
the whole day through
Ah, one look in my eyes
and you can tell that's true
\end{Chorus}
\espaceInterStrophe

\begin{Verse}
Old man look at my life
I'm a lot like you were
\bis
\end{Verse}
\vfill
~
\end{multicols}

\vfill

\begin{Chords}[Verse]
\hline
D & F\majsept & C & G\\\hline
D & F\majsept & C & F\majsept\\\hline
D & F\majsept & C & G\\\hline
D & C & F & G\\\hline
\end{Chords}
\espaceInterGrille

\begin{Chords}[Chorus]
\hline
D(\sus2) & D(\sus4) & \slashbox{A\mineur\sept}{A\mineur\sept\ E\mineur\sept} & \slashbox{E\mineur\sept}{E\mineur\sept\ G}\\\hline
\end{Chords}

\vfill

\end{Song}



