\begin{Song}{Hit The Road Jack}{Ray Charles}[Percy Mayfield]

\begin{Paratext}
This song was written by rhythm and bluesman Percy Mayfield and first recorded in 1960 as an \emph{a cappella} demo sent to Art Rupe. It became famous after it was recorded by singer-songwriter-pianist Ray Charles with the Raelettes vocalist Margie Hendricks.

The song, which has a strong beat, is a brief, rather comic duet between a fed-up woman and her good-for-nothing man. He tries to wheedle her into letting him stay, but she will have none of it, "'cause it's understood: you ain't got no money, you just ain't no good."
\end{Paratext}
\espaceParatexteVersMultiColonnes

\begin{multicols}{2}
\begin{Chorus}
Hit the road Jack and don't you come back
No more, no more, no more, no more
Hit the road Jack and don't you come back no more
What you say?
Hit the road Jack and don't you come back
No more, no more, no more, no more
Hit the road Jack and don't you come back no more
\end{Chorus}
\espaceInterStrophe

\begin{Verse}
Woah, woman, oh woman, don't treat me so mean
You're the meanest old woman that I've ever seen
I guess if you said so
I'd have to pack my things and go
(That's right)
\end{Verse}
\espaceInterStrophe

\tochorus
\espaceInterStrophe

\begin{Verse}
Now baby, listen baby, don't ya treat me this a-way
'Cause I'll be back on my feet some day
Don't care if you do 'cause it's understood
You ain't got no money you just ain't no good
Well, I guess if you say so
I'd have to pack my things and go
(That's right)
\end{Verse}
\espaceInterStrophe

\tochorus
\espaceInterStrophe
\columnbreak

\begin{Chorus}
Well
(Don't you come back no more)
Uh, what you say?
(Don't you come back no more)
\espaceInterStrophe

I didn't understand you
(Don't you come back no more)
You can't mean that
(Don't you come back no more)
\espaceInterStrophe

Oh, now baby, please
(Don't you come back no more)
What you tryin' to do to me?
(Don't you come back no more)
\espaceInterStrophe

Oh, don't treat me like that
(Don't you come back no more)
\end{Chorus}

\end{multicols}

\vfill

\begin{multicols}{2}

\begin{Chords}[Originale Piano/Cuivres]
\hline
A\bemol\mineur\add6 & F\diese & E\sept & D\diese\sept\\\hline
\end{Chords}
\espaceInterGrille

\begin{Chords}[Adaptée Guitare]
\hline
A\mineur\add6 & G & F\sept & E\sept\\\hline
\end{Chords}

\end{multicols}

\vfill

\end{Song}



