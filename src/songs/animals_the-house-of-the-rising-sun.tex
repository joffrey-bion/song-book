\begin{Song}[House of the Rising Sun]{The House of the Rising Sun}[Animals]{The Animals}[Inconnu]

\begin{Paratext}
Cette chanson est une ballade folk des Etats-Unis, dont on ne connaît pas
vraiment l'auteur original. Le plus viel enregistrement connu est celui de
\emph{Tom Clarence Ashley} et \emph{Gwen Foster}, datant de 1934. Cette vieille
version est un blues en \emph{E}.
Ce serait \emph{Dave Van Ronk} qui serait à l'origine de la version moderne (en
\emph{Am}) en 1964, que \emph{Bob Dylan} et les \emph{Animals} ont repris par
la suite.
\end{Paratext}
\espaceParatexteVersSimpleColonne

\begin{Verse}
There is a house in New Orleans
They call the Rising Sun
And it's been the ruin of many a poor boy
And God I know I'm one
\espaceInterStrophe

My mother was a tailor
She sewed my new blue jeans
My father was a gamblin' man
Down in New Orleans
\espaceInterStrophe

Now the only thing a gambler needs
Is a suitcase and trunk
And the only time he's satisfied
Is when he's on a drunk
\end{Verse}
\espaceInterStrophe

\direction{Organ solo}\\
\espaceInterStrophe

\begin{Verse}
Oh mother tell your children
Not to do what I have done
Spend your lives in sin and misery
In the House of the Rising Sun
\espaceInterStrophe

Well, I got one foot on the platform
The other foot on the train
I'm goin' back to New Orleans
To wear that ball and chain
\espaceInterStrophe

Well, there is a house in New Orleans
They call the Rising Sun
And it's been the ruin of many a poor boy
And God I know I'm one
\end{Verse}

\vfill
\begin{Chords}[Verse]
\hline
A\mineur & C & D        & F      \\\hline
A\mineur & C & E        & E\sept \\\hline
A\mineur & C & D        & F      \\\hline
A\mineur & E & A\mineur & E\sept \\\hline
\end{Chords}
\vfill
\end{Song}


